% Options for packages loaded elsewhere
\PassOptionsToPackage{unicode}{hyperref}
\PassOptionsToPackage{hyphens}{url}
%
\documentclass[
]{article}
\usepackage{amsmath,amssymb}
\usepackage{lmodern}
\usepackage{iftex}
\ifPDFTeX
  \usepackage[T1]{fontenc}
  \usepackage[utf8]{inputenc}
  \usepackage{textcomp} % provide euro and other symbols
\else % if luatex or xetex
  \usepackage{unicode-math}
  \defaultfontfeatures{Scale=MatchLowercase}
  \defaultfontfeatures[\rmfamily]{Ligatures=TeX,Scale=1}
\fi
% Use upquote if available, for straight quotes in verbatim environments
\IfFileExists{upquote.sty}{\usepackage{upquote}}{}
\IfFileExists{microtype.sty}{% use microtype if available
  \usepackage[]{microtype}
  \UseMicrotypeSet[protrusion]{basicmath} % disable protrusion for tt fonts
}{}
\makeatletter
\@ifundefined{KOMAClassName}{% if non-KOMA class
  \IfFileExists{parskip.sty}{%
    \usepackage{parskip}
  }{% else
    \setlength{\parindent}{0pt}
    \setlength{\parskip}{6pt plus 2pt minus 1pt}}
}{% if KOMA class
  \KOMAoptions{parskip=half}}
\makeatother
\usepackage{xcolor}
\usepackage[margin=1in]{geometry}
\usepackage{color}
\usepackage{fancyvrb}
\newcommand{\VerbBar}{|}
\newcommand{\VERB}{\Verb[commandchars=\\\{\}]}
\DefineVerbatimEnvironment{Highlighting}{Verbatim}{commandchars=\\\{\}}
% Add ',fontsize=\small' for more characters per line
\usepackage{framed}
\definecolor{shadecolor}{RGB}{248,248,248}
\newenvironment{Shaded}{\begin{snugshade}}{\end{snugshade}}
\newcommand{\AlertTok}[1]{\textcolor[rgb]{0.94,0.16,0.16}{#1}}
\newcommand{\AnnotationTok}[1]{\textcolor[rgb]{0.56,0.35,0.01}{\textbf{\textit{#1}}}}
\newcommand{\AttributeTok}[1]{\textcolor[rgb]{0.77,0.63,0.00}{#1}}
\newcommand{\BaseNTok}[1]{\textcolor[rgb]{0.00,0.00,0.81}{#1}}
\newcommand{\BuiltInTok}[1]{#1}
\newcommand{\CharTok}[1]{\textcolor[rgb]{0.31,0.60,0.02}{#1}}
\newcommand{\CommentTok}[1]{\textcolor[rgb]{0.56,0.35,0.01}{\textit{#1}}}
\newcommand{\CommentVarTok}[1]{\textcolor[rgb]{0.56,0.35,0.01}{\textbf{\textit{#1}}}}
\newcommand{\ConstantTok}[1]{\textcolor[rgb]{0.00,0.00,0.00}{#1}}
\newcommand{\ControlFlowTok}[1]{\textcolor[rgb]{0.13,0.29,0.53}{\textbf{#1}}}
\newcommand{\DataTypeTok}[1]{\textcolor[rgb]{0.13,0.29,0.53}{#1}}
\newcommand{\DecValTok}[1]{\textcolor[rgb]{0.00,0.00,0.81}{#1}}
\newcommand{\DocumentationTok}[1]{\textcolor[rgb]{0.56,0.35,0.01}{\textbf{\textit{#1}}}}
\newcommand{\ErrorTok}[1]{\textcolor[rgb]{0.64,0.00,0.00}{\textbf{#1}}}
\newcommand{\ExtensionTok}[1]{#1}
\newcommand{\FloatTok}[1]{\textcolor[rgb]{0.00,0.00,0.81}{#1}}
\newcommand{\FunctionTok}[1]{\textcolor[rgb]{0.00,0.00,0.00}{#1}}
\newcommand{\ImportTok}[1]{#1}
\newcommand{\InformationTok}[1]{\textcolor[rgb]{0.56,0.35,0.01}{\textbf{\textit{#1}}}}
\newcommand{\KeywordTok}[1]{\textcolor[rgb]{0.13,0.29,0.53}{\textbf{#1}}}
\newcommand{\NormalTok}[1]{#1}
\newcommand{\OperatorTok}[1]{\textcolor[rgb]{0.81,0.36,0.00}{\textbf{#1}}}
\newcommand{\OtherTok}[1]{\textcolor[rgb]{0.56,0.35,0.01}{#1}}
\newcommand{\PreprocessorTok}[1]{\textcolor[rgb]{0.56,0.35,0.01}{\textit{#1}}}
\newcommand{\RegionMarkerTok}[1]{#1}
\newcommand{\SpecialCharTok}[1]{\textcolor[rgb]{0.00,0.00,0.00}{#1}}
\newcommand{\SpecialStringTok}[1]{\textcolor[rgb]{0.31,0.60,0.02}{#1}}
\newcommand{\StringTok}[1]{\textcolor[rgb]{0.31,0.60,0.02}{#1}}
\newcommand{\VariableTok}[1]{\textcolor[rgb]{0.00,0.00,0.00}{#1}}
\newcommand{\VerbatimStringTok}[1]{\textcolor[rgb]{0.31,0.60,0.02}{#1}}
\newcommand{\WarningTok}[1]{\textcolor[rgb]{0.56,0.35,0.01}{\textbf{\textit{#1}}}}
\usepackage{graphicx}
\makeatletter
\def\maxwidth{\ifdim\Gin@nat@width>\linewidth\linewidth\else\Gin@nat@width\fi}
\def\maxheight{\ifdim\Gin@nat@height>\textheight\textheight\else\Gin@nat@height\fi}
\makeatother
% Scale images if necessary, so that they will not overflow the page
% margins by default, and it is still possible to overwrite the defaults
% using explicit options in \includegraphics[width, height, ...]{}
\setkeys{Gin}{width=\maxwidth,height=\maxheight,keepaspectratio}
% Set default figure placement to htbp
\makeatletter
\def\fps@figure{htbp}
\makeatother
\setlength{\emergencystretch}{3em} % prevent overfull lines
\providecommand{\tightlist}{%
  \setlength{\itemsep}{0pt}\setlength{\parskip}{0pt}}
\setcounter{secnumdepth}{-\maxdimen} % remove section numbering
\ifLuaTeX
  \usepackage{selnolig}  % disable illegal ligatures
\fi
\IfFileExists{bookmark.sty}{\usepackage{bookmark}}{\usepackage{hyperref}}
\IfFileExists{xurl.sty}{\usepackage{xurl}}{} % add URL line breaks if available
\urlstyle{same} % disable monospaced font for URLs
\hypersetup{
  pdftitle={Bellabeat Case Study},
  pdfauthor={Bishop Sally},
  hidelinks,
  pdfcreator={LaTeX via pandoc}}

\title{Bellabeat Case Study}
\author{Bishop Sally}
\date{2023-06-01}

\begin{document}
\maketitle

\hypertarget{introduction}{%
\section{\texorpdfstring{\textbf{Introduction}}{Introduction}}\label{introduction}}

This is my breakdown of my Bellabeat Case study capstone project, which
is part of my Google Data Analytics Certificate. This breakdown will
give general insights to the analyst team at Bellabeat to make informed
business decisions that can help the company in the future The method I
used to analyze Bellabeat was breaking everything down by using:

ASK PREPARE PROCESS ANALYZE SHARE ACT

\hypertarget{about-bellabeat}{%
\subsection{\texorpdfstring{\textbf{About
Bellabeat}}{About Bellabeat}}\label{about-bellabeat}}

\href{https://bellabeat.com/?gclid=CjwKCAjwg-GjBhBnEiwAMUvNW8dU3gg_7xVeMp_WV-1Acp6kc_57Kj7kPTRdLJCLFy2Cq3PRWln7cRoCep0QAvD_BwE}{Bellabeat}
is a high-tech manufacturer of health-focused products for women.
Bellabeat is a successful small company, but they have the potential to
become a larger player in the global smart device market. Urška Sršen,
cofounder and Chief Creative Officer of Bellabeat, believes that
analyzing smart device fitness data could help unlock new growth
opportunities for the company.

\hypertarget{ask}{%
\section{\texorpdfstring{\textbf{Ask}}{Ask}}\label{ask}}

In the \emph{Ask} section we should ask some key questions to find the
problem and how we can solve them. The key questions that need to be
asked to gain further insight are: \textbf{1.} What are the trends of
smart device usuage? \textbf{2.} How can these trends apply to Bellabeat
Customers? \textbf{3.} How can these trends help influence Bellabeat
Marketing Strategy

\textbf{Business Task:} The questions that need to be asked will lead us
to analyze smart device usage to gain insight into how consumers use
non-Bellabeat smart devices. After we gain the insights we need we will
then select one Bellabeat product that we can apply to these insights.

\textbf{How I will report the information:} 1. Give a clear summary of
the business task 2. Adding a description of all data sources used 3.
Documenting any cleaning or manipulation of data 4. Providing a summary
of my analysis 5. Supporting visualizations and key findings 6. Your top
high-level content recommendations based on your analysis

\textbf{Identifying Key Stakeholders:} \emph{\textbf{Primary
StakeHolders:} - \textbf{Urška Sršen:} Bellabeat's cofounder and Chief
Creative Officer - \textbf{Sando Mur:} Mathematician and Bellabeat's
cofounder; key member of the Bellabeat executive team Secondary
Stakeholders: }\_\_Secondary Stakeholders:\_\_ - \textbf{Bellabeat
marketing analytics team:} A team of data analysts responsible for
collecting, analyzing, and reporting data that helps guide Bellabeat's
marketing strategy

\hypertarget{prepare}{%
\section{\texorpdfstring{\textbf{Prepare}}{Prepare}}\label{prepare}}

In the prepare section we find, download, and store the data in a
secured folder.

\begin{itemize}
\tightlist
\item
  \textbf{What data was gathered:}
  \href{https://www.kaggle.com/datasets/arashnic/fitbit}{*Kaggle
  Dataset}

  \begin{itemize}
  \tightlist
  \item
    This data set was generated by respondents to a distributed survey
    via \href{https://zenodo.org/record/53894\#.YMoUpnVKiP9}{Amazon
    Mechanical Turk} between 03.12.2016-05.12.2016. Thirty eligible
    Fitbit users consented to the submission of personal tracker data,
    including minute-level output for physical activity, heart rate, and
    sleep monitoring.
  \end{itemize}
\item
  \textbf{Data Summary}:

  \begin{itemize}
  \tightlist
  \item
    33 Fitbit users agreed to submit personal data while using their
    Fitbit Trackers.The data tracked contained different factors for
    each data set
  \item
    There is a total of 18 csv files that were used for the
    \href{https://www.kaggle.com/datasets/arashnic/fitbit}{FitBit
    Fitness Tracker Data}. Out of the 18, I used 8 out of the 18 data
    sets to help me find a solution. These data sets included:
    \emph{daily\_activity}, \emph{daily\_calories},
    \emph{daily\_intensities}, \emph{daily\_steps},
    \emph{hourly\_calories}, \emph{hourly\_steps},\emph{sleep\_day}, and
    \emph{weight\_info}.
  \item
    The data was generated between March 12 2016 through May 12 2016
  \item
    Source: Furberg, R., Brinton, J., Keating, M., \& Ortiz, A. (2016).
    Crowd-sourced Fitbit data sets 03.12.2016-05.12.2016 {[}Data set{]}.
    Zenodo. \url{https://doi.org/10.5281/zenodo.53894}
  \item
    \href{https://creativecommons.org/licenses/by/4.0/legalcode}{Creative
    Commons Atrribution 4.0 International}
  \end{itemize}
\end{itemize}

\hypertarget{data-reliability}{%
\subsection{\texorpdfstring{\textbf{Data
Reliability}}{Data Reliability}}\label{data-reliability}}

\begin{itemize}
\tightlist
\item
  33 participants is considered a sufficient sample size, but a larger
  sample size would be more efficient based on the market size of the
  fitness tracker market as a whole.
\item
  The participants are also anonymous so it does not include specific
  demographic data to avoid any biases.
\item
  The data tracked activity and correlation but it does not state any
  specific features being utilized and tracked during this study.
\item
  The data was generated between March and May 2016, so this data can be
  considered old and not being up to date with newer trends and features
  of Fitness Trackers
\end{itemize}

\hypertarget{conclusion-i-think-overall-this-data-is-sufficient-enough-to-get-a-general-consensus-of-the-fitness-tracker-market-but-a-larger-and-more-current-dataset-would-be-a-good-alternative-in-the-future.}{%
\subsection{\texorpdfstring{\textbf{Conclusion:} I think overall this
data is sufficient enough to get a general consensus of the Fitness
tracker market, but a larger and more current dataset would be a good
alternative in the
future.}{Conclusion: I think overall this data is sufficient enough to get a general consensus of the Fitness tracker market, but a larger and more current dataset would be a good alternative in the future.}}\label{conclusion-i-think-overall-this-data-is-sufficient-enough-to-get-a-general-consensus-of-the-fitness-tracker-market-but-a-larger-and-more-current-dataset-would-be-a-good-alternative-in-the-future.}}

\hypertarget{process}{%
\section{\texorpdfstring{\textbf{Process}}{Process}}\label{process}}

After securing the data now we can load, view, and clean the data to
make sure the data is valid enough to complete the remaining steps

\begin{Shaded}
\begin{Highlighting}[]
\CommentTok{\# Installing necessary packages}
\FunctionTok{options}\NormalTok{(}\AttributeTok{repos =} \StringTok{"https://cran.rstudio.com/"}\NormalTok{)  }\CommentTok{\# Set the CRAN mirror}

\FunctionTok{install.packages}\NormalTok{(}\StringTok{"tidyverse"}\NormalTok{)}
\end{Highlighting}
\end{Shaded}

\begin{verbatim}
## Installing package into 'C:/Users/bsall/AppData/Local/R/win-library/4.3'
## (as 'lib' is unspecified)
\end{verbatim}

\begin{verbatim}
## package 'tidyverse' successfully unpacked and MD5 sums checked
## 
## The downloaded binary packages are in
##  C:\Users\bsall\AppData\Local\Temp\RtmpQzJs7T\downloaded_packages
\end{verbatim}

\begin{Shaded}
\begin{Highlighting}[]
\FunctionTok{install.packages}\NormalTok{(}\StringTok{"ggplot2"}\NormalTok{)}
\end{Highlighting}
\end{Shaded}

\begin{verbatim}
## Installing package into 'C:/Users/bsall/AppData/Local/R/win-library/4.3'
## (as 'lib' is unspecified)
\end{verbatim}

\begin{verbatim}
## package 'ggplot2' successfully unpacked and MD5 sums checked
## 
## The downloaded binary packages are in
##  C:\Users\bsall\AppData\Local\Temp\RtmpQzJs7T\downloaded_packages
\end{verbatim}

\begin{Shaded}
\begin{Highlighting}[]
\FunctionTok{install.packages}\NormalTok{(}\StringTok{"janitor"}\NormalTok{)}
\end{Highlighting}
\end{Shaded}

\begin{verbatim}
## Installing package into 'C:/Users/bsall/AppData/Local/R/win-library/4.3'
## (as 'lib' is unspecified)
\end{verbatim}

\begin{verbatim}
## package 'janitor' successfully unpacked and MD5 sums checked
## 
## The downloaded binary packages are in
##  C:\Users\bsall\AppData\Local\Temp\RtmpQzJs7T\downloaded_packages
\end{verbatim}

\begin{Shaded}
\begin{Highlighting}[]
\CommentTok{\#Loading necessary packages }

\FunctionTok{library}\NormalTok{(tidyverse)}
\end{Highlighting}
\end{Shaded}

\begin{verbatim}
## -- Attaching core tidyverse packages ------------------------ tidyverse 2.0.0 --
## v dplyr     1.1.2     v readr     2.1.4
## v forcats   1.0.0     v stringr   1.5.0
## v ggplot2   3.4.2     v tibble    3.2.1
## v lubridate 1.9.2     v tidyr     1.3.0
## v purrr     1.0.1     
## -- Conflicts ------------------------------------------ tidyverse_conflicts() --
## x dplyr::filter() masks stats::filter()
## x dplyr::lag()    masks stats::lag()
## i Use the conflicted package (<http://conflicted.r-lib.org/>) to force all conflicts to become errors
\end{verbatim}

\begin{Shaded}
\begin{Highlighting}[]
\FunctionTok{library}\NormalTok{(ggplot2)}
\FunctionTok{library}\NormalTok{(janitor)}
\end{Highlighting}
\end{Shaded}

\begin{verbatim}
## 
## Attaching package: 'janitor'
## 
## The following objects are masked from 'package:stats':
## 
##     chisq.test, fisher.test
\end{verbatim}

\begin{Shaded}
\begin{Highlighting}[]
\FunctionTok{library}\NormalTok{(readr)}
\end{Highlighting}
\end{Shaded}

\hypertarget{importing-data}{%
\subsection{\texorpdfstring{\textbf{Importing
Data}}{Importing Data}}\label{importing-data}}

\begin{Shaded}
\begin{Highlighting}[]
\CommentTok{\#Loading csv files into data sets }

\NormalTok{daily\_activity }\OtherTok{\textless{}{-}} \FunctionTok{read\_csv}\NormalTok{(}\StringTok{"C:/Users/bsall/OneDrive/Documents/Fitabase Data 4.12.16{-}5.12.16/dailyActivity\_merged.csv"}\NormalTok{)}
\end{Highlighting}
\end{Shaded}

\begin{verbatim}
## Rows: 940 Columns: 15
## -- Column specification --------------------------------------------------------
## Delimiter: ","
## chr  (1): ActivityDate
## dbl (14): Id, TotalSteps, TotalDistance, TrackerDistance, LoggedActivitiesDi...
## 
## i Use `spec()` to retrieve the full column specification for this data.
## i Specify the column types or set `show_col_types = FALSE` to quiet this message.
\end{verbatim}

\begin{Shaded}
\begin{Highlighting}[]
\FunctionTok{View}\NormalTok{(daily\_activity)}

\NormalTok{daily\_calories }\OtherTok{\textless{}{-}} \FunctionTok{read\_csv}\NormalTok{(}\StringTok{"C:/Users/bsall/OneDrive/Documents/Fitabase Data 4.12.16{-}5.12.16/dailyCalories\_merged.csv"}\NormalTok{)}
\end{Highlighting}
\end{Shaded}

\begin{verbatim}
## Rows: 940 Columns: 3
## -- Column specification --------------------------------------------------------
## Delimiter: ","
## chr (1): ActivityDay
## dbl (2): Id, Calories
## 
## i Use `spec()` to retrieve the full column specification for this data.
## i Specify the column types or set `show_col_types = FALSE` to quiet this message.
\end{verbatim}

\begin{Shaded}
\begin{Highlighting}[]
\FunctionTok{View}\NormalTok{(daily\_calories)}

\NormalTok{daily\_intensities }\OtherTok{\textless{}{-}} \FunctionTok{read\_csv}\NormalTok{(}\StringTok{"C:/Users/bsall/OneDrive/Documents/Fitabase Data 4.12.16{-}5.12.16/dailyIntensities\_merged.csv"}\NormalTok{)}
\end{Highlighting}
\end{Shaded}

\begin{verbatim}
## Rows: 940 Columns: 10
## -- Column specification --------------------------------------------------------
## Delimiter: ","
## chr (1): ActivityDay
## dbl (9): Id, SedentaryMinutes, LightlyActiveMinutes, FairlyActiveMinutes, Ve...
## 
## i Use `spec()` to retrieve the full column specification for this data.
## i Specify the column types or set `show_col_types = FALSE` to quiet this message.
\end{verbatim}

\begin{Shaded}
\begin{Highlighting}[]
\FunctionTok{View}\NormalTok{(daily\_intensities)}

\NormalTok{daily\_steps }\OtherTok{\textless{}{-}} \FunctionTok{read\_csv}\NormalTok{(}\StringTok{"C:/Users/bsall/OneDrive/Documents/Fitabase Data 4.12.16{-}5.12.16/dailySteps\_merged.csv"}\NormalTok{)}
\end{Highlighting}
\end{Shaded}

\begin{verbatim}
## Rows: 940 Columns: 3
## -- Column specification --------------------------------------------------------
## Delimiter: ","
## chr (1): ActivityDay
## dbl (2): Id, StepTotal
## 
## i Use `spec()` to retrieve the full column specification for this data.
## i Specify the column types or set `show_col_types = FALSE` to quiet this message.
\end{verbatim}

\begin{Shaded}
\begin{Highlighting}[]
\FunctionTok{View}\NormalTok{(daily\_steps)}

\NormalTok{sleep\_day }\OtherTok{\textless{}{-}} \FunctionTok{read\_csv}\NormalTok{(}\StringTok{"C:/Users/bsall/OneDrive/Documents/Fitabase Data 4.12.16{-}5.12.16/sleepDay\_merged.csv"}\NormalTok{)}
\end{Highlighting}
\end{Shaded}

\begin{verbatim}
## Rows: 413 Columns: 5
## -- Column specification --------------------------------------------------------
## Delimiter: ","
## chr (1): SleepDay
## dbl (4): Id, TotalSleepRecords, TotalMinutesAsleep, TotalTimeInBed
## 
## i Use `spec()` to retrieve the full column specification for this data.
## i Specify the column types or set `show_col_types = FALSE` to quiet this message.
\end{verbatim}

\begin{Shaded}
\begin{Highlighting}[]
\FunctionTok{View}\NormalTok{(sleep\_day)}

\NormalTok{weight\_info }\OtherTok{\textless{}{-}} \FunctionTok{read\_csv}\NormalTok{(}\StringTok{"C:/Users/bsall/OneDrive/Documents/Fitabase Data 4.12.16{-}5.12.16/weightLogInfo\_merged.csv"}\NormalTok{)}
\end{Highlighting}
\end{Shaded}

\begin{verbatim}
## Rows: 67 Columns: 8
## -- Column specification --------------------------------------------------------
## Delimiter: ","
## chr (1): Date
## dbl (6): Id, WeightKg, WeightPounds, Fat, BMI, LogId
## lgl (1): IsManualReport
## 
## i Use `spec()` to retrieve the full column specification for this data.
## i Specify the column types or set `show_col_types = FALSE` to quiet this message.
\end{verbatim}

\begin{Shaded}
\begin{Highlighting}[]
\FunctionTok{View}\NormalTok{(weight\_info)}

\NormalTok{hourly\_steps }\OtherTok{\textless{}{-}} \FunctionTok{read\_csv}\NormalTok{(}\StringTok{"C:/Users/bsall/OneDrive/Documents/Fitabase Data 4.12.16{-}5.12.16/hourlySteps\_merged.csv"}\NormalTok{)}
\end{Highlighting}
\end{Shaded}

\begin{verbatim}
## Rows: 22099 Columns: 3
## -- Column specification --------------------------------------------------------
## Delimiter: ","
## chr (1): ActivityHour
## dbl (2): Id, StepTotal
## 
## i Use `spec()` to retrieve the full column specification for this data.
## i Specify the column types or set `show_col_types = FALSE` to quiet this message.
\end{verbatim}

\begin{Shaded}
\begin{Highlighting}[]
\FunctionTok{View}\NormalTok{(hourly\_steps)}

\NormalTok{hourly\_Calories }\OtherTok{\textless{}{-}} \FunctionTok{read\_csv}\NormalTok{(}\StringTok{"C:/Users/bsall/OneDrive/Documents/Fitabase Data 4.12.16{-}5.12.16/hourlyCalories\_merged.csv"}\NormalTok{)}
\end{Highlighting}
\end{Shaded}

\begin{verbatim}
## Rows: 22099 Columns: 3
## -- Column specification --------------------------------------------------------
## Delimiter: ","
## chr (1): ActivityHour
## dbl (2): Id, Calories
## 
## i Use `spec()` to retrieve the full column specification for this data.
## i Specify the column types or set `show_col_types = FALSE` to quiet this message.
\end{verbatim}

\begin{Shaded}
\begin{Highlighting}[]
\FunctionTok{View}\NormalTok{(hourly\_Calories)}
\end{Highlighting}
\end{Shaded}

\begin{Shaded}
\begin{Highlighting}[]
\CommentTok{\# Removing unnecessary columns}

\NormalTok{daily\_activity }\OtherTok{=} \FunctionTok{subset}\NormalTok{(daily\_activity,}\AttributeTok{select =} \SpecialCharTok{{-}}\FunctionTok{c}\NormalTok{(LoggedActivitiesDistance, VeryActiveDistance, ModeratelyActiveDistance, LightActiveDistance, SedentaryActiveDistance))}

\NormalTok{daily\_intensities }\OtherTok{=} \FunctionTok{subset}\NormalTok{(daily\_intensities,}\AttributeTok{select =} \SpecialCharTok{{-}}\FunctionTok{c}\NormalTok{(SedentaryActiveDistance, LightActiveDistance, ModeratelyActiveDistance, VeryActiveDistance))}
\end{Highlighting}
\end{Shaded}

\begin{Shaded}
\begin{Highlighting}[]
\CommentTok{\# Finding null or missing values in my datasets}

\FunctionTok{is.null}\NormalTok{(daily\_activity)}
\end{Highlighting}
\end{Shaded}

\begin{verbatim}
## [1] FALSE
\end{verbatim}

\begin{Shaded}
\begin{Highlighting}[]
\FunctionTok{is.null}\NormalTok{(daily\_calories)}
\end{Highlighting}
\end{Shaded}

\begin{verbatim}
## [1] FALSE
\end{verbatim}

\begin{Shaded}
\begin{Highlighting}[]
\FunctionTok{is.null}\NormalTok{(daily\_intensities)}
\end{Highlighting}
\end{Shaded}

\begin{verbatim}
## [1] FALSE
\end{verbatim}

\begin{Shaded}
\begin{Highlighting}[]
\FunctionTok{is.null}\NormalTok{(daily\_steps)}
\end{Highlighting}
\end{Shaded}

\begin{verbatim}
## [1] FALSE
\end{verbatim}

\begin{Shaded}
\begin{Highlighting}[]
\FunctionTok{is.null}\NormalTok{ (sleep\_day)}
\end{Highlighting}
\end{Shaded}

\begin{verbatim}
## [1] FALSE
\end{verbatim}

\begin{Shaded}
\begin{Highlighting}[]
\FunctionTok{is.null}\NormalTok{(weight\_info)}
\end{Highlighting}
\end{Shaded}

\begin{verbatim}
## [1] FALSE
\end{verbatim}

\begin{Shaded}
\begin{Highlighting}[]
\FunctionTok{is.null}\NormalTok{(hourly\_Calories)}
\end{Highlighting}
\end{Shaded}

\begin{verbatim}
## [1] FALSE
\end{verbatim}

\begin{Shaded}
\begin{Highlighting}[]
\FunctionTok{is.null}\NormalTok{(hourly\_steps)}
\end{Highlighting}
\end{Shaded}

\begin{verbatim}
## [1] FALSE
\end{verbatim}

\begin{Shaded}
\begin{Highlighting}[]
\CommentTok{\#Viewing the columns to check for any inconsistent data }

\FunctionTok{head}\NormalTok{(daily\_activity)}
\end{Highlighting}
\end{Shaded}

\begin{verbatim}
## # A tibble: 6 x 10
##       Id ActivityDate TotalSteps TotalDistance TrackerDistance VeryActiveMinutes
##    <dbl> <chr>             <dbl>         <dbl>           <dbl>             <dbl>
## 1 1.50e9 4/12/2016         13162          8.5             8.5                 25
## 2 1.50e9 4/13/2016         10735          6.97            6.97                21
## 3 1.50e9 4/14/2016         10460          6.74            6.74                30
## 4 1.50e9 4/15/2016          9762          6.28            6.28                29
## 5 1.50e9 4/16/2016         12669          8.16            8.16                36
## 6 1.50e9 4/17/2016          9705          6.48            6.48                38
## # i 4 more variables: FairlyActiveMinutes <dbl>, LightlyActiveMinutes <dbl>,
## #   SedentaryMinutes <dbl>, Calories <dbl>
\end{verbatim}

\begin{Shaded}
\begin{Highlighting}[]
\FunctionTok{str}\NormalTok{(daily\_activity)}
\end{Highlighting}
\end{Shaded}

\begin{verbatim}
## tibble [940 x 10] (S3: tbl_df/tbl/data.frame)
##  $ Id                  : num [1:940] 1.5e+09 1.5e+09 1.5e+09 1.5e+09 1.5e+09 ...
##  $ ActivityDate        : chr [1:940] "4/12/2016" "4/13/2016" "4/14/2016" "4/15/2016" ...
##  $ TotalSteps          : num [1:940] 13162 10735 10460 9762 12669 ...
##  $ TotalDistance       : num [1:940] 8.5 6.97 6.74 6.28 8.16 ...
##  $ TrackerDistance     : num [1:940] 8.5 6.97 6.74 6.28 8.16 ...
##  $ VeryActiveMinutes   : num [1:940] 25 21 30 29 36 38 42 50 28 19 ...
##  $ FairlyActiveMinutes : num [1:940] 13 19 11 34 10 20 16 31 12 8 ...
##  $ LightlyActiveMinutes: num [1:940] 328 217 181 209 221 164 233 264 205 211 ...
##  $ SedentaryMinutes    : num [1:940] 728 776 1218 726 773 ...
##  $ Calories            : num [1:940] 1985 1797 1776 1745 1863 ...
\end{verbatim}

\begin{Shaded}
\begin{Highlighting}[]
\FunctionTok{n\_distinct}\NormalTok{(sleep\_day}\SpecialCharTok{$}\NormalTok{Id)}
\end{Highlighting}
\end{Shaded}

\begin{verbatim}
## [1] 24
\end{verbatim}

\begin{Shaded}
\begin{Highlighting}[]
\FunctionTok{n\_distinct}\NormalTok{(daily\_activity}\SpecialCharTok{$}\NormalTok{Id)}
\end{Highlighting}
\end{Shaded}

\begin{verbatim}
## [1] 33
\end{verbatim}

\begin{Shaded}
\begin{Highlighting}[]
\FunctionTok{is\_empty}\NormalTok{(daily\_activity)}
\end{Highlighting}
\end{Shaded}

\begin{verbatim}
## [1] FALSE
\end{verbatim}

\begin{Shaded}
\begin{Highlighting}[]
\FunctionTok{is\_empty}\NormalTok{(daily\_calories)}
\end{Highlighting}
\end{Shaded}

\begin{verbatim}
## [1] FALSE
\end{verbatim}

\begin{Shaded}
\begin{Highlighting}[]
\FunctionTok{is\_empty}\NormalTok{(daily\_intensities)}
\end{Highlighting}
\end{Shaded}

\begin{verbatim}
## [1] FALSE
\end{verbatim}

\begin{Shaded}
\begin{Highlighting}[]
\FunctionTok{is\_empty}\NormalTok{(daily\_steps)}
\end{Highlighting}
\end{Shaded}

\begin{verbatim}
## [1] FALSE
\end{verbatim}

\begin{Shaded}
\begin{Highlighting}[]
\FunctionTok{is\_empty}\NormalTok{(sleep\_day)}
\end{Highlighting}
\end{Shaded}

\begin{verbatim}
## [1] FALSE
\end{verbatim}

\begin{Shaded}
\begin{Highlighting}[]
\FunctionTok{is\_empty}\NormalTok{(weight\_info)}
\end{Highlighting}
\end{Shaded}

\begin{verbatim}
## [1] FALSE
\end{verbatim}

\begin{Shaded}
\begin{Highlighting}[]
\FunctionTok{is\_empty}\NormalTok{(hourly\_Calories)}
\end{Highlighting}
\end{Shaded}

\begin{verbatim}
## [1] FALSE
\end{verbatim}

\begin{Shaded}
\begin{Highlighting}[]
\FunctionTok{is\_empty}\NormalTok{(hourly\_steps)}
\end{Highlighting}
\end{Shaded}

\begin{verbatim}
## [1] FALSE
\end{verbatim}

\begin{Shaded}
\begin{Highlighting}[]
\CommentTok{\# Updating and reformatting the date columns to get a more desired format }

\NormalTok{daily\_activity}\SpecialCharTok{$}\NormalTok{ActivityDate }\OtherTok{\textless{}{-}} \FunctionTok{as.Date}\NormalTok{(daily\_activity}\SpecialCharTok{$}\NormalTok{ActivityDate, }\StringTok{"\%m/\%d/\%y"}\NormalTok{)}
\NormalTok{daily\_calories}\SpecialCharTok{$}\NormalTok{ActivityDay }\OtherTok{\textless{}{-}} \FunctionTok{as.Date}\NormalTok{(daily\_calories}\SpecialCharTok{$}\NormalTok{ActivityDay, }\StringTok{"\%m/\%d/\%y"}\NormalTok{)}
\NormalTok{daily\_intensities}\SpecialCharTok{$}\NormalTok{ActivityDay }\OtherTok{\textless{}{-}} \FunctionTok{as.Date}\NormalTok{(daily\_intensities}\SpecialCharTok{$}\NormalTok{ActivityDay, }\StringTok{"\%m/\%d/\%y"}\NormalTok{)}
\NormalTok{daily\_steps}\SpecialCharTok{$}\NormalTok{ActivityDay }\OtherTok{\textless{}{-}} \FunctionTok{as.Date}\NormalTok{(daily\_steps}\SpecialCharTok{$}\NormalTok{ActivityDay, }\StringTok{"\%m/\%d/\%y"}\NormalTok{)}
\NormalTok{sleep\_day}\SpecialCharTok{$}\NormalTok{SleepDay }\OtherTok{\textless{}{-}} \FunctionTok{as.Date}\NormalTok{(}\FunctionTok{strptime}\NormalTok{(sleep\_day}\SpecialCharTok{$}\NormalTok{SleepDay, }\StringTok{"\%m/\%d/\%Y"}\NormalTok{))}
\NormalTok{hourly\_Calories}\SpecialCharTok{$}\NormalTok{ActivityHour }\OtherTok{\textless{}{-}} \FunctionTok{as.Date}\NormalTok{(hourly\_Calories}\SpecialCharTok{$}\NormalTok{ActivityHour, }\StringTok{"\%m/\%d/\%y"}\NormalTok{)}
\NormalTok{hourly\_steps}\SpecialCharTok{$}\NormalTok{ActivityHour }\OtherTok{\textless{}{-}} \FunctionTok{as.Date}\NormalTok{(hourly\_steps}\SpecialCharTok{$}\NormalTok{ActivityHour, }\StringTok{"\%m/\%d/\%y"}\NormalTok{)}
\NormalTok{weight\_info}\SpecialCharTok{$}\NormalTok{Date }\OtherTok{\textless{}{-}} \FunctionTok{as.Date}\NormalTok{(weight\_info}\SpecialCharTok{$}\NormalTok{Date, }\StringTok{"\%m/\%d/\%y"}\NormalTok{)}
\end{Highlighting}
\end{Shaded}

\begin{Shaded}
\begin{Highlighting}[]
\CommentTok{\# Checking for the new date format }

\FunctionTok{str}\NormalTok{(daily\_activity)}
\end{Highlighting}
\end{Shaded}

\begin{verbatim}
## tibble [940 x 10] (S3: tbl_df/tbl/data.frame)
##  $ Id                  : num [1:940] 1.5e+09 1.5e+09 1.5e+09 1.5e+09 1.5e+09 ...
##  $ ActivityDate        : Date[1:940], format: "2020-04-12" "2020-04-13" ...
##  $ TotalSteps          : num [1:940] 13162 10735 10460 9762 12669 ...
##  $ TotalDistance       : num [1:940] 8.5 6.97 6.74 6.28 8.16 ...
##  $ TrackerDistance     : num [1:940] 8.5 6.97 6.74 6.28 8.16 ...
##  $ VeryActiveMinutes   : num [1:940] 25 21 30 29 36 38 42 50 28 19 ...
##  $ FairlyActiveMinutes : num [1:940] 13 19 11 34 10 20 16 31 12 8 ...
##  $ LightlyActiveMinutes: num [1:940] 328 217 181 209 221 164 233 264 205 211 ...
##  $ SedentaryMinutes    : num [1:940] 728 776 1218 726 773 ...
##  $ Calories            : num [1:940] 1985 1797 1776 1745 1863 ...
\end{verbatim}

\begin{Shaded}
\begin{Highlighting}[]
\FunctionTok{str}\NormalTok{(daily\_calories)}
\end{Highlighting}
\end{Shaded}

\begin{verbatim}
## spc_tbl_ [940 x 3] (S3: spec_tbl_df/tbl_df/tbl/data.frame)
##  $ Id         : num [1:940] 1.5e+09 1.5e+09 1.5e+09 1.5e+09 1.5e+09 ...
##  $ ActivityDay: Date[1:940], format: "2020-04-12" "2020-04-13" ...
##  $ Calories   : num [1:940] 1985 1797 1776 1745 1863 ...
##  - attr(*, "spec")=
##   .. cols(
##   ..   Id = col_double(),
##   ..   ActivityDay = col_character(),
##   ..   Calories = col_double()
##   .. )
##  - attr(*, "problems")=<externalptr>
\end{verbatim}

\begin{Shaded}
\begin{Highlighting}[]
\FunctionTok{str}\NormalTok{(daily\_steps)}
\end{Highlighting}
\end{Shaded}

\begin{verbatim}
## spc_tbl_ [940 x 3] (S3: spec_tbl_df/tbl_df/tbl/data.frame)
##  $ Id         : num [1:940] 1.5e+09 1.5e+09 1.5e+09 1.5e+09 1.5e+09 ...
##  $ ActivityDay: Date[1:940], format: "2020-04-12" "2020-04-13" ...
##  $ StepTotal  : num [1:940] 13162 10735 10460 9762 12669 ...
##  - attr(*, "spec")=
##   .. cols(
##   ..   Id = col_double(),
##   ..   ActivityDay = col_character(),
##   ..   StepTotal = col_double()
##   .. )
##  - attr(*, "problems")=<externalptr>
\end{verbatim}

\begin{Shaded}
\begin{Highlighting}[]
\FunctionTok{str}\NormalTok{(sleep\_day)}
\end{Highlighting}
\end{Shaded}

\begin{verbatim}
## spc_tbl_ [413 x 5] (S3: spec_tbl_df/tbl_df/tbl/data.frame)
##  $ Id                : num [1:413] 1.5e+09 1.5e+09 1.5e+09 1.5e+09 1.5e+09 ...
##  $ SleepDay          : Date[1:413], format: "2016-04-12" "2016-04-13" ...
##  $ TotalSleepRecords : num [1:413] 1 2 1 2 1 1 1 1 1 1 ...
##  $ TotalMinutesAsleep: num [1:413] 327 384 412 340 700 304 360 325 361 430 ...
##  $ TotalTimeInBed    : num [1:413] 346 407 442 367 712 320 377 364 384 449 ...
##  - attr(*, "spec")=
##   .. cols(
##   ..   Id = col_double(),
##   ..   SleepDay = col_character(),
##   ..   TotalSleepRecords = col_double(),
##   ..   TotalMinutesAsleep = col_double(),
##   ..   TotalTimeInBed = col_double()
##   .. )
##  - attr(*, "problems")=<externalptr>
\end{verbatim}

\begin{Shaded}
\begin{Highlighting}[]
\FunctionTok{str}\NormalTok{(daily\_intensities)}
\end{Highlighting}
\end{Shaded}

\begin{verbatim}
## tibble [940 x 6] (S3: tbl_df/tbl/data.frame)
##  $ Id                  : num [1:940] 1.5e+09 1.5e+09 1.5e+09 1.5e+09 1.5e+09 ...
##  $ ActivityDay         : Date[1:940], format: "2020-04-12" "2020-04-13" ...
##  $ SedentaryMinutes    : num [1:940] 728 776 1218 726 773 ...
##  $ LightlyActiveMinutes: num [1:940] 328 217 181 209 221 164 233 264 205 211 ...
##  $ FairlyActiveMinutes : num [1:940] 13 19 11 34 10 20 16 31 12 8 ...
##  $ VeryActiveMinutes   : num [1:940] 25 21 30 29 36 38 42 50 28 19 ...
\end{verbatim}

\begin{Shaded}
\begin{Highlighting}[]
\FunctionTok{str}\NormalTok{(hourly\_Calories)}
\end{Highlighting}
\end{Shaded}

\begin{verbatim}
## spc_tbl_ [22,099 x 3] (S3: spec_tbl_df/tbl_df/tbl/data.frame)
##  $ Id          : num [1:22099] 1.5e+09 1.5e+09 1.5e+09 1.5e+09 1.5e+09 ...
##  $ ActivityHour: Date[1:22099], format: "2020-04-12" "2020-04-12" ...
##  $ Calories    : num [1:22099] 81 61 59 47 48 48 48 47 68 141 ...
##  - attr(*, "spec")=
##   .. cols(
##   ..   Id = col_double(),
##   ..   ActivityHour = col_character(),
##   ..   Calories = col_double()
##   .. )
##  - attr(*, "problems")=<externalptr>
\end{verbatim}

\begin{Shaded}
\begin{Highlighting}[]
\FunctionTok{str}\NormalTok{(hourly\_steps)}
\end{Highlighting}
\end{Shaded}

\begin{verbatim}
## spc_tbl_ [22,099 x 3] (S3: spec_tbl_df/tbl_df/tbl/data.frame)
##  $ Id          : num [1:22099] 1.5e+09 1.5e+09 1.5e+09 1.5e+09 1.5e+09 ...
##  $ ActivityHour: Date[1:22099], format: "2020-04-12" "2020-04-12" ...
##  $ StepTotal   : num [1:22099] 373 160 151 0 0 ...
##  - attr(*, "spec")=
##   .. cols(
##   ..   Id = col_double(),
##   ..   ActivityHour = col_character(),
##   ..   StepTotal = col_double()
##   .. )
##  - attr(*, "problems")=<externalptr>
\end{verbatim}

\begin{Shaded}
\begin{Highlighting}[]
\FunctionTok{str}\NormalTok{(weight\_info)}
\end{Highlighting}
\end{Shaded}

\begin{verbatim}
## spc_tbl_ [67 x 8] (S3: spec_tbl_df/tbl_df/tbl/data.frame)
##  $ Id            : num [1:67] 1.50e+09 1.50e+09 1.93e+09 2.87e+09 2.87e+09 ...
##  $ Date          : Date[1:67], format: "2020-05-02" "2020-05-03" ...
##  $ WeightKg      : num [1:67] 52.6 52.6 133.5 56.7 57.3 ...
##  $ WeightPounds  : num [1:67] 116 116 294 125 126 ...
##  $ Fat           : num [1:67] 22 NA NA NA NA 25 NA NA NA NA ...
##  $ BMI           : num [1:67] 22.6 22.6 47.5 21.5 21.7 ...
##  $ IsManualReport: logi [1:67] TRUE TRUE FALSE TRUE TRUE TRUE ...
##  $ LogId         : num [1:67] 1.46e+12 1.46e+12 1.46e+12 1.46e+12 1.46e+12 ...
##  - attr(*, "spec")=
##   .. cols(
##   ..   Id = col_double(),
##   ..   Date = col_character(),
##   ..   WeightKg = col_double(),
##   ..   WeightPounds = col_double(),
##   ..   Fat = col_double(),
##   ..   BMI = col_double(),
##   ..   IsManualReport = col_logical(),
##   ..   LogId = col_double()
##   .. )
##  - attr(*, "problems")=<externalptr>
\end{verbatim}

\begin{Shaded}
\begin{Highlighting}[]
\CommentTok{\# merging data sets: sleep\_day and hourly\_steps to create *combined\_bars*}

\NormalTok{combined\_bars }\OtherTok{\textless{}{-}} \FunctionTok{merge}\NormalTok{(sleep\_day, hourly\_steps, }\AttributeTok{by =} \FunctionTok{c}\NormalTok{(}\StringTok{\textquotesingle{}Id\textquotesingle{}}\NormalTok{))}

\CommentTok{\# viewing the newly merged data }

\FunctionTok{view}\NormalTok{(combined\_bars)}
\FunctionTok{head}\NormalTok{(combined\_bars)}
\end{Highlighting}
\end{Shaded}

\begin{verbatim}
##           Id   SleepDay TotalSleepRecords TotalMinutesAsleep TotalTimeInBed
## 1 1503960366 2016-04-12                 1                327            346
## 2 1503960366 2016-04-12                 1                327            346
## 3 1503960366 2016-04-12                 1                327            346
## 4 1503960366 2016-04-12                 1                327            346
## 5 1503960366 2016-04-12                 1                327            346
## 6 1503960366 2016-04-12                 1                327            346
##   ActivityHour StepTotal
## 1   2020-04-12      1864
## 2   2020-04-12      1166
## 3   2020-04-12       676
## 4   2020-04-14         0
## 5   2020-04-12       344
## 6   2020-04-12       250
\end{verbatim}

\begin{Shaded}
\begin{Highlighting}[]
\CommentTok{\# Merging the data sets: daily\_activity and daily\_steps to create *record\_activity*}

\NormalTok{record\_activity }\OtherTok{\textless{}{-}} \FunctionTok{merge}\NormalTok{(daily\_activity, daily\_steps, }\AttributeTok{by =} \FunctionTok{c}\NormalTok{(}\StringTok{\textquotesingle{}Id\textquotesingle{}}\NormalTok{))}

\CommentTok{\#Viewing the newly merged data set}

\FunctionTok{view}\NormalTok{(record\_activity)}
\FunctionTok{head}\NormalTok{(record\_activity)}
\end{Highlighting}
\end{Shaded}

\begin{verbatim}
##           Id ActivityDate TotalSteps TotalDistance TrackerDistance
## 1 1503960366   2020-04-12      13162           8.5             8.5
## 2 1503960366   2020-04-12      13162           8.5             8.5
## 3 1503960366   2020-04-12      13162           8.5             8.5
## 4 1503960366   2020-04-12      13162           8.5             8.5
## 5 1503960366   2020-04-12      13162           8.5             8.5
## 6 1503960366   2020-04-12      13162           8.5             8.5
##   VeryActiveMinutes FairlyActiveMinutes LightlyActiveMinutes SedentaryMinutes
## 1                25                  13                  328              728
## 2                25                  13                  328              728
## 3                25                  13                  328              728
## 4                25                  13                  328              728
## 5                25                  13                  328              728
## 6                25                  13                  328              728
##   Calories ActivityDay StepTotal
## 1     1985  2020-04-12     13162
## 2     1985  2020-04-13     10735
## 3     1985  2020-04-14     10460
## 4     1985  2020-04-15      9762
## 5     1985  2020-04-16     12669
## 6     1985  2020-04-17      9705
\end{verbatim}

\begin{Shaded}
\begin{Highlighting}[]
\CommentTok{\# Mutating all the time based columns to create a total minutes column that combines the times together }

\NormalTok{total\_intensities }\OtherTok{\textless{}{-}}\NormalTok{ daily\_intensities }\SpecialCharTok{\%\textgreater{}\%}
  \FunctionTok{mutate}\NormalTok{(}\AttributeTok{daily\_intensities =} \FunctionTok{rowSums}\NormalTok{(}\FunctionTok{select}\NormalTok{(., SedentaryMinutes, LightlyActiveMinutes, FairlyActiveMinutes, VeryActiveMinutes), }\AttributeTok{na.rm =} \ConstantTok{TRUE}\NormalTok{))}
\end{Highlighting}
\end{Shaded}

\begin{Shaded}
\begin{Highlighting}[]
\CommentTok{\#This string adds the calories column from the daily\_calories data set into the total\_intensities data set }

\NormalTok{total\_intensities}\SpecialCharTok{$}\NormalTok{Calories }\OtherTok{\textless{}{-}}\NormalTok{ daily\_calories}\SpecialCharTok{$}\NormalTok{Calories}

\CommentTok{\#This changes the colname of daily\_intensities into *TotalMinutes* to help organize the colnames when entering the data}

\FunctionTok{names}\NormalTok{(total\_intensities)[}\FunctionTok{names}\NormalTok{(total\_intensities) }\SpecialCharTok{==} \StringTok{"daily\_intensities"}\NormalTok{] }\OtherTok{\textless{}{-}} \StringTok{"TotalMinutes"}
\end{Highlighting}
\end{Shaded}

\begin{Shaded}
\begin{Highlighting}[]
\CommentTok{\# Removes all falsely reported data sets that were presented in the IsManualReport column }
\NormalTok{weight\_info }\OtherTok{\textless{}{-}} \FunctionTok{subset}\NormalTok{(weight\_info, IsManualReport }\SpecialCharTok{==} \ConstantTok{TRUE}\NormalTok{)}
\end{Highlighting}
\end{Shaded}

\begin{Shaded}
\begin{Highlighting}[]
\CommentTok{\#Summarizing weight of the selected columns and merged the data into the new variable *weight\_summary* }

\NormalTok{weight\_summary }\OtherTok{\textless{}{-}}\NormalTok{ weight\_info }\SpecialCharTok{\%\textgreater{}\%}
  \FunctionTok{group\_by}\NormalTok{(Date) }\SpecialCharTok{\%\textgreater{}\%}
  \FunctionTok{summarise}\NormalTok{(}\AttributeTok{AverageWeight =} \FunctionTok{mean}\NormalTok{(WeightPounds))}

\FunctionTok{glimpse}\NormalTok{(weight\_summary}\SpecialCharTok{$}\NormalTok{AverageWeight)}
\end{Highlighting}
\end{Shaded}

\begin{verbatim}
##  num [1:30] 138 137 136 136 137 ...
\end{verbatim}

\begin{Shaded}
\begin{Highlighting}[]
\FunctionTok{glimpse}\NormalTok{(weight\_info}\SpecialCharTok{$}\NormalTok{WeightPounds)}
\end{Highlighting}
\end{Shaded}

\begin{verbatim}
##  num [1:41] 116 116 125 126 160 ...
\end{verbatim}

\begin{Shaded}
\begin{Highlighting}[]
\FunctionTok{glimpse}\NormalTok{(weight\_info}\SpecialCharTok{$}\NormalTok{Id)}
\end{Highlighting}
\end{Shaded}

\begin{verbatim}
##  num [1:41] 1.50e+09 1.50e+09 2.87e+09 2.87e+09 4.32e+09 ...
\end{verbatim}

\hypertarget{analyze}{%
\section{\texorpdfstring{\textbf{Analyze}}{Analyze}}\label{analyze}}

After the data is cleaned now we can analyze the data so we can get an
idea of how we are going to create and share the data

\begin{Shaded}
\begin{Highlighting}[]
\CommentTok{\# Getting an overall summary of the data present in the *daily\_activity* data set before I begin visualizing }
\NormalTok{daily\_activity }\SpecialCharTok{\%\textgreater{}\%}
  \FunctionTok{select}\NormalTok{(TotalSteps,}
\NormalTok{         TotalDistance, TrackerDistance, VeryActiveMinutes, FairlyActiveMinutes, LightlyActiveMinutes, SedentaryMinutes, Calories) }\SpecialCharTok{\%\textgreater{}\%}
  \FunctionTok{summary}\NormalTok{()}
\end{Highlighting}
\end{Shaded}

\begin{verbatim}
##    TotalSteps    TotalDistance    TrackerDistance  VeryActiveMinutes
##  Min.   :    0   Min.   : 0.000   Min.   : 0.000   Min.   :  0.00   
##  1st Qu.: 3790   1st Qu.: 2.620   1st Qu.: 2.620   1st Qu.:  0.00   
##  Median : 7406   Median : 5.245   Median : 5.245   Median :  4.00   
##  Mean   : 7638   Mean   : 5.490   Mean   : 5.475   Mean   : 21.16   
##  3rd Qu.:10727   3rd Qu.: 7.713   3rd Qu.: 7.710   3rd Qu.: 32.00   
##  Max.   :36019   Max.   :28.030   Max.   :28.030   Max.   :210.00   
##  FairlyActiveMinutes LightlyActiveMinutes SedentaryMinutes    Calories   
##  Min.   :  0.00      Min.   :  0.0        Min.   :   0.0   Min.   :   0  
##  1st Qu.:  0.00      1st Qu.:127.0        1st Qu.: 729.8   1st Qu.:1828  
##  Median :  6.00      Median :199.0        Median :1057.5   Median :2134  
##  Mean   : 13.56      Mean   :192.8        Mean   : 991.2   Mean   :2304  
##  3rd Qu.: 19.00      3rd Qu.:264.0        3rd Qu.:1229.5   3rd Qu.:2793  
##  Max.   :143.00      Max.   :518.0        Max.   :1440.0   Max.   :4900
\end{verbatim}

\begin{Shaded}
\begin{Highlighting}[]
\CommentTok{\#Summarizing *sleep* and *Time* based columns in the **sleep\_day** variable }
\NormalTok{sleep\_day }\SpecialCharTok{\%\textgreater{}\%}
  \FunctionTok{select}\NormalTok{(TotalSleepRecords, TotalMinutesAsleep, TotalTimeInBed) }\SpecialCharTok{\%\textgreater{}\%}
  \FunctionTok{summary}\NormalTok{() }
\end{Highlighting}
\end{Shaded}

\begin{verbatim}
##  TotalSleepRecords TotalMinutesAsleep TotalTimeInBed 
##  Min.   :1.000     Min.   : 58.0      Min.   : 61.0  
##  1st Qu.:1.000     1st Qu.:361.0      1st Qu.:403.0  
##  Median :1.000     Median :433.0      Median :463.0  
##  Mean   :1.119     Mean   :419.5      Mean   :458.6  
##  3rd Qu.:1.000     3rd Qu.:490.0      3rd Qu.:526.0  
##  Max.   :3.000     Max.   :796.0      Max.   :961.0
\end{verbatim}

\begin{Shaded}
\begin{Highlighting}[]
\CommentTok{\#Summarizing the *ActivityDate* and  *StepTotal* columns in the **record\_activity** variable }

\NormalTok{record\_activity }\SpecialCharTok{\%\textgreater{}\%} 
  \FunctionTok{select}\NormalTok{(ActivityDate, StepTotal) }\SpecialCharTok{\%\textgreater{}\%}
  \FunctionTok{summary}\NormalTok{()}
\end{Highlighting}
\end{Shaded}

\begin{verbatim}
##   ActivityDate          StepTotal    
##  Min.   :2020-04-12   Min.   :    0  
##  1st Qu.:2020-04-19   1st Qu.: 3761  
##  Median :2020-04-26   Median : 7443  
##  Mean   :2020-04-26   Mean   : 7673  
##  3rd Qu.:2020-05-04   3rd Qu.:10771  
##  Max.   :2020-05-12   Max.   :36019
\end{verbatim}

\begin{Shaded}
\begin{Highlighting}[]
\CommentTok{\#Summarizing the *TotalMinutes* and *Calories* columns in the **total\_intensities** variable }

\NormalTok{total\_intensities }\SpecialCharTok{\%\textgreater{}\%}
  \FunctionTok{select}\NormalTok{(TotalMinutes,Calories) }\SpecialCharTok{\%\textgreater{}\%}
  \FunctionTok{summary}\NormalTok{()}
\end{Highlighting}
\end{Shaded}

\begin{verbatim}
##   TotalMinutes       Calories   
##  Min.   :   2.0   Min.   :   0  
##  1st Qu.: 989.8   1st Qu.:1828  
##  Median :1440.0   Median :2134  
##  Mean   :1218.8   Mean   :2304  
##  3rd Qu.:1440.0   3rd Qu.:2793  
##  Max.   :1440.0   Max.   :4900
\end{verbatim}

\begin{Shaded}
\begin{Highlighting}[]
\CommentTok{\#Summarizing the *ActivityDate* and *StepTotal* columns in the **record\_activity** variable }

\NormalTok{record\_activity }\SpecialCharTok{\%\textgreater{}\%} 
  \FunctionTok{select}\NormalTok{(ActivityDate, StepTotal) }\SpecialCharTok{\%\textgreater{}\%}
  \FunctionTok{summary}\NormalTok{()}
\end{Highlighting}
\end{Shaded}

\begin{verbatim}
##   ActivityDate          StepTotal    
##  Min.   :2020-04-12   Min.   :    0  
##  1st Qu.:2020-04-19   1st Qu.: 3761  
##  Median :2020-04-26   Median : 7443  
##  Mean   :2020-04-26   Mean   : 7673  
##  3rd Qu.:2020-05-04   3rd Qu.:10771  
##  Max.   :2020-05-12   Max.   :36019
\end{verbatim}

\begin{Shaded}
\begin{Highlighting}[]
\CommentTok{\#Summarizing the *Date* and *AverageWeight* columns in the **weight\_summary** variable }

\NormalTok{weight\_summary }\SpecialCharTok{\%\textgreater{}\%}
  \FunctionTok{select}\NormalTok{(Date,AverageWeight) }\SpecialCharTok{\%\textgreater{}\%}
  \FunctionTok{summary}\NormalTok{()}
\end{Highlighting}
\end{Shaded}

\begin{verbatim}
##       Date            AverageWeight  
##  Min.   :2020-04-12   Min.   :125.2  
##  1st Qu.:2020-04-19   1st Qu.:134.9  
##  Median :2020-04-27   Median :135.6  
##  Mean   :2020-04-27   Mean   :137.0  
##  3rd Qu.:2020-05-04   3rd Qu.:136.9  
##  Max.   :2020-05-12   Max.   :147.5
\end{verbatim}

\hypertarget{share}{%
\section{\texorpdfstring{\textbf{Share}}{Share}}\label{share}}

Now that we analyzed the different data sets now we can share the
insights gathered to bring visuals to the important data shown

\begin{Shaded}
\begin{Highlighting}[]
\CommentTok{\#This visual shows the relationship between Calories burned and Total Steps tracked }

\FunctionTok{ggplot}\NormalTok{(}\AttributeTok{data =}\NormalTok{ daily\_activity)}\SpecialCharTok{+}
  \FunctionTok{geom\_point}\NormalTok{(}\AttributeTok{mapping =} \FunctionTok{aes}\NormalTok{(}\AttributeTok{x =}\NormalTok{ TotalSteps, }\AttributeTok{y =}\NormalTok{ Calories, }\AttributeTok{color =}\NormalTok{ Calories)) }\SpecialCharTok{+}
  \FunctionTok{geom\_smooth}\NormalTok{(}\AttributeTok{mapping =} \FunctionTok{aes}\NormalTok{(}\AttributeTok{x =}\NormalTok{ TotalSteps, }\AttributeTok{y =}\NormalTok{ Calories))}\SpecialCharTok{+} 
  \FunctionTok{ggtitle}\NormalTok{(}\StringTok{"Calories vs. Total Steps"}\NormalTok{)}\SpecialCharTok{+}
  \FunctionTok{labs}\NormalTok{(}\AttributeTok{title =} \StringTok{"Calories vs. Total Steps"}\NormalTok{,}
       \AttributeTok{x =} \StringTok{"Total Steps"}\NormalTok{, }\AttributeTok{y =} \StringTok{"Calories"}\NormalTok{)}
\end{Highlighting}
\end{Shaded}

\begin{verbatim}
## `geom_smooth()` using method = 'loess' and formula = 'y ~ x'
\end{verbatim}

\includegraphics{Capstone_Markdown_files/figure-latex/unnamed-chunk-22-1.pdf}

\begin{Shaded}
\begin{Highlighting}[]
\CommentTok{\# Shows the relationship between Calories burned and Total Distance of the exercise activities }

\FunctionTok{ggplot}\NormalTok{(}\AttributeTok{data =}\NormalTok{ daily\_activity)}\SpecialCharTok{+}
  \FunctionTok{geom\_point}\NormalTok{(}\AttributeTok{mapping =} \FunctionTok{aes}\NormalTok{(}\AttributeTok{x =}\NormalTok{ TotalDistance, }\AttributeTok{y =}\NormalTok{ Calories, }\AttributeTok{color =}\NormalTok{ Calories))}\SpecialCharTok{+}
  \FunctionTok{geom\_smooth}\NormalTok{(}\AttributeTok{mapping =} \FunctionTok{aes}\NormalTok{(}\AttributeTok{x =}\NormalTok{ TotalDistance, }\AttributeTok{y =}\NormalTok{ Calories))}\SpecialCharTok{+}
  \FunctionTok{ggtitle}\NormalTok{(}\StringTok{"Calories vs. Total Distance"}\NormalTok{)}\SpecialCharTok{+}
  \FunctionTok{labs}\NormalTok{(}\AttributeTok{title =} \StringTok{"Calories vs. Total Distance"}\NormalTok{,}
       \AttributeTok{x =} \StringTok{"Total Distance"}\NormalTok{, }\AttributeTok{y =} \StringTok{"Calories"}\NormalTok{)}
\end{Highlighting}
\end{Shaded}

\begin{verbatim}
## `geom_smooth()` using method = 'loess' and formula = 'y ~ x'
\end{verbatim}

\includegraphics{Capstone_Markdown_files/figure-latex/unnamed-chunk-23-1.pdf}

\begin{Shaded}
\begin{Highlighting}[]
\CommentTok{\# Shows the relationship between the users *Time Spent in Bed* vs. *Total Steps* }

\FunctionTok{ggplot}\NormalTok{(}\AttributeTok{data =}\NormalTok{ combined\_bars) }\SpecialCharTok{+}
  \FunctionTok{geom\_bar}\NormalTok{(}\AttributeTok{mapping =} \FunctionTok{aes}\NormalTok{(}\AttributeTok{x =}\NormalTok{ TotalMinutesAsleep), }\AttributeTok{stat =} \StringTok{"count"}\NormalTok{, }\AttributeTok{color =} \StringTok{"steelblue"}\NormalTok{, }\AttributeTok{fill =} \StringTok{"white"}\NormalTok{) }\SpecialCharTok{+}
  \FunctionTok{labs}\NormalTok{(}\AttributeTok{x =} \StringTok{"Time in Bed"}\NormalTok{, }\AttributeTok{y =} \StringTok{"Steps"}\NormalTok{, }\AttributeTok{title =} \StringTok{"Rest vs. Steps"}\NormalTok{) }\SpecialCharTok{+}
  \FunctionTok{theme\_minimal}\NormalTok{() }\SpecialCharTok{+}
  \FunctionTok{theme}\NormalTok{(}\AttributeTok{plot.title =} \FunctionTok{element\_text}\NormalTok{(}\AttributeTok{size =} \DecValTok{18}\NormalTok{, }\AttributeTok{face =} \StringTok{"bold"}\NormalTok{),}
        \AttributeTok{axis.title =} \FunctionTok{element\_text}\NormalTok{(}\AttributeTok{size =} \DecValTok{14}\NormalTok{),}
        \AttributeTok{axis.text =} \FunctionTok{element\_text}\NormalTok{(}\AttributeTok{size =} \DecValTok{12}\NormalTok{),}
        \AttributeTok{panel.grid.major =} \FunctionTok{element\_line}\NormalTok{(}\AttributeTok{color =} \StringTok{"gray80"}\NormalTok{),}
        \AttributeTok{panel.grid.minor =} \FunctionTok{element\_blank}\NormalTok{(),}
        \AttributeTok{panel.background =} \FunctionTok{element\_rect}\NormalTok{(}\AttributeTok{fill =} \StringTok{"white"}\NormalTok{))}
\end{Highlighting}
\end{Shaded}

\includegraphics{Capstone_Markdown_files/figure-latex/unnamed-chunk-24-1.pdf}

\begin{Shaded}
\begin{Highlighting}[]
\CommentTok{\# This graph shows the **rise** of the users steps over time }

\FunctionTok{ggplot}\NormalTok{(record\_activity, }\FunctionTok{aes}\NormalTok{(}\AttributeTok{x =}\NormalTok{ ActivityDate, }\AttributeTok{y =}\NormalTok{ StepTotal)) }\SpecialCharTok{+}
  \FunctionTok{geom\_smooth}\NormalTok{() }\SpecialCharTok{+}
  \FunctionTok{labs}\NormalTok{(}\AttributeTok{x =} \StringTok{"Days Recorded"}\NormalTok{, }\AttributeTok{y =} \StringTok{"Steps Taken"}\NormalTok{, }\AttributeTok{title =} \StringTok{"Days Recorded vs. Steps Taken"}\NormalTok{)}
\end{Highlighting}
\end{Shaded}

\begin{verbatim}
## `geom_smooth()` using method = 'gam' and formula = 'y ~ s(x, bs = "cs")'
\end{verbatim}

\includegraphics{Capstone_Markdown_files/figure-latex/unnamed-chunk-25-1.pdf}

\begin{Shaded}
\begin{Highlighting}[]
\CommentTok{\# This graph shows the correlation between *longer exercise* leading to *more calories burned* }

\FunctionTok{ggplot}\NormalTok{(total\_intensities, }\FunctionTok{aes}\NormalTok{(}\AttributeTok{x =}\NormalTok{ TotalMinutes, }\AttributeTok{y =}\NormalTok{ Calories))}\SpecialCharTok{+}
  \FunctionTok{geom\_smooth}\NormalTok{()}\SpecialCharTok{+}
  \FunctionTok{labs}\NormalTok{(}\AttributeTok{x =} \StringTok{"Total Minutes"}\NormalTok{, }\AttributeTok{y =} \StringTok{"Calories Burned"}\NormalTok{, }\AttributeTok{title =} \StringTok{"Intensity vs Calories Burned"}\NormalTok{)}
\end{Highlighting}
\end{Shaded}

\begin{verbatim}
## `geom_smooth()` using method = 'loess' and formula = 'y ~ x'
\end{verbatim}

\includegraphics{Capstone_Markdown_files/figure-latex/unnamed-chunk-26-1.pdf}

\begin{Shaded}
\begin{Highlighting}[]
\CommentTok{\# This graph shows the amount of *calories burned hourly each day* between all the participants }

\FunctionTok{ggplot}\NormalTok{(}\AttributeTok{data =}\NormalTok{ hourly\_Calories, }\FunctionTok{aes}\NormalTok{(}\AttributeTok{x =}\NormalTok{ ActivityHour, }\AttributeTok{y =}\NormalTok{ Calories))}\SpecialCharTok{+}
  \FunctionTok{geom\_line}\NormalTok{(}\AttributeTok{size =} \FloatTok{1.5}\NormalTok{)}\SpecialCharTok{+}
  \FunctionTok{labs}\NormalTok{(}\AttributeTok{title =} \StringTok{"Calories Burned by Day"}\NormalTok{)}
\end{Highlighting}
\end{Shaded}

\begin{verbatim}
## Warning: Using `size` aesthetic for lines was deprecated in ggplot2 3.4.0.
## i Please use `linewidth` instead.
## This warning is displayed once every 8 hours.
## Call `lifecycle::last_lifecycle_warnings()` to see where this warning was
## generated.
\end{verbatim}

\includegraphics{Capstone_Markdown_files/figure-latex/unnamed-chunk-27-1.pdf}

\begin{Shaded}
\begin{Highlighting}[]
\CommentTok{\# Shows the amount of *steps taken hourly each day* between all the participants }

\FunctionTok{ggplot}\NormalTok{(}\AttributeTok{data =}\NormalTok{ hourly\_steps, }\FunctionTok{aes}\NormalTok{(}\AttributeTok{x =}\NormalTok{ ActivityHour, }\AttributeTok{y =}\NormalTok{ StepTotal))}\SpecialCharTok{+}
  \FunctionTok{geom\_line}\NormalTok{(}\AttributeTok{size =} \FloatTok{1.5}\NormalTok{)}\SpecialCharTok{+}
  \FunctionTok{labs}\NormalTok{(}\AttributeTok{title =} \StringTok{"Hourly Steps by Day"}\NormalTok{)}
\end{Highlighting}
\end{Shaded}

\includegraphics{Capstone_Markdown_files/figure-latex/unnamed-chunk-28-1.pdf}

\begin{Shaded}
\begin{Highlighting}[]
\CommentTok{\# This graph shows the *participants weight from all of the values that were TRUE*}

\FunctionTok{ggplot}\NormalTok{(}\AttributeTok{data =}\NormalTok{ weight\_info, }\FunctionTok{aes}\NormalTok{(}\AttributeTok{x =}\NormalTok{ Id, }\AttributeTok{y =}\NormalTok{ WeightPounds)) }\SpecialCharTok{+}
  \FunctionTok{geom\_step}\NormalTok{(}\AttributeTok{size =} \DecValTok{2}\NormalTok{, }\AttributeTok{color =} \StringTok{"steelblue"}\NormalTok{) }\SpecialCharTok{+}
  \FunctionTok{labs}\NormalTok{(}\AttributeTok{title =} \StringTok{"Participant Weight"}\NormalTok{) }\SpecialCharTok{+}
  \FunctionTok{theme\_minimal}\NormalTok{()}
\end{Highlighting}
\end{Shaded}

\includegraphics{Capstone_Markdown_files/figure-latex/unnamed-chunk-29-1.pdf}

\begin{Shaded}
\begin{Highlighting}[]
\CommentTok{\# Visualizes the *weight averages for each day recorded* }

\FunctionTok{ggplot}\NormalTok{(}\AttributeTok{data =}\NormalTok{ weight\_summary, }\FunctionTok{aes}\NormalTok{(}\AttributeTok{x =}\NormalTok{ Date, }\AttributeTok{y =}\NormalTok{ AverageWeight)) }\SpecialCharTok{+}
  \FunctionTok{geom\_col}\NormalTok{(}\AttributeTok{fill =} \StringTok{"steelblue"}\NormalTok{) }\SpecialCharTok{+}
  \FunctionTok{labs}\NormalTok{(}\AttributeTok{title =} \StringTok{"Average Weight by Day"}\NormalTok{) }\SpecialCharTok{+} 
  \FunctionTok{scale\_x\_date}\NormalTok{(}\AttributeTok{date\_labels =} \StringTok{"\%b \%d, \%Y"}\NormalTok{, }\AttributeTok{date\_breaks =} \StringTok{"1 week"}\NormalTok{)}
\end{Highlighting}
\end{Shaded}

\includegraphics{Capstone_Markdown_files/figure-latex/unnamed-chunk-30-1.pdf}

\hypertarget{act}{%
\section{\texorpdfstring{\textbf{Act}}{Act}}\label{act}}

\hypertarget{product-focus-bellabeat-app}{%
\subsection{\texorpdfstring{\textbf{Product focus: Bellabeat
app}}{Product focus: Bellabeat app}}\label{product-focus-bellabeat-app}}

\begin{itemize}
\item
  By analyzing the data above I came to the conclusion that as time goes
  on there is more \emph{activity} and \emph{usage}, but towards end
  there seems to be a downward slope of each graph, which are both shown
  in the \textbf{Calories vs.~Total Distance} and \textbf{Calories
  vs.~Total Steps}. This means after a month there should be more
  incentives and interact-able features to keep user retention on the
  app
\item
  \emph{Sleep} and \emph{activity} seem to be more consistent in the
  middle levels of the chart which is shown in the \textbf{Rest
  vs.~Steps} chart, so I think its important for users to have at least
  \textbf{400-500 minutes} \emph{6-8 hours} of sleep each day. Data like
  this shows the opportunity to implement systems such as \emph{sleep
  trackers}, \emph{meal plans}, \emph{fitness guides}, \emph{interactive
  profiles} and many other features to keep the users returning to the
  app on a consistent basis
\item
  Also, reviewing the \emph{weight} of each user, we can see the users
  have a vast difference in weight between each other as shown in the
  \textbf{Participant Weight} and \textbf{Average Weight by Day} graphs.
  This means that each user is going to have unique goals so it's
  important to cater to all kinds of users by considering weight and
  other attributes to retain usability of the smart devices and app.
\end{itemize}

\hypertarget{conclusion}{%
\subsection{\texorpdfstring{\textbf{Conclusion}}{Conclusion}}\label{conclusion}}

Overall, I think Bellabeat has a great opportunity to innovate and add
helpful and informative features to the app to help retain user traffic
on the app so they can help reach their goals, as well as learning and
helping others along the way by including an interactive interface to
the app. With the Fitness tracker market contentiously growing with
large growth rate potential in the future as well, there are many ways
\emph{Bellabeat} can take advantage of this opportunity to innovate by
being unique compared to other competitors. As \emph{Bellabeat} focuses
on innovation they can also focus heavily on marketing to reach a larger
audience of consumers that highlights the importance using the
\emph{app} to help reach personal goals and to help others along the
way.

\end{document}
